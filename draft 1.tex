\documentclass[english]{article}
\usepackage[T1]{fontenc}
%\usepackage[latin9]{inputenc}
%\usepackage{geometry}
%\geometry{verbose,tmargin=1in,bmargin=1in,lmargin=1in,rmargin=1in}
\usepackage{amsmath}
\usepackage{amssymb}
\usepackage{setspace}
\usepackage[utf8]{inputenc}
\usepackage{graphicx}
\usepackage{float}
\usepackage{adjustbox}
\usepackage{gensymb}
\usepackage{amssymb}
\usepackage{array}
\usepackage{ragged2e}
\usepackage{lipsum}
\onehalfspacing
\usepackage{babel}
\usepackage{ctable}
\usepackage{booktabs}
\usepackage{graphicx}
\usepackage{caption}
\usepackage{subcaption}
\usepackage{placeins}
\usepackage{todonotes}

\begin{document}
\title{Informality in LAC: Heterogeneity and Disappointing Progress}
\maketitle
\begin{abstract}
    This descriptive paper uses household and employment surveys from the Latin America and the Caribbean region to paint a more complete picture of the different aspects of informality. We start by discussing alternative informality measures at the regional and cross country level. We then show that there has been progress on increasing the share of dependent workers who contribute to social security but that the productive structure of the economies in the region has remained mostly unchanged. We argue that this is mainly coming from the focus of governments policies on facilitating the registration of low productivity workers, therefore subsidizing low productivity firms. To make this point we use two approaches: a qualitative approach based on the analysis of policies that have been identified as successful in reducing informality. Second, we use a simple decomposition on different formalization margins to show that the formalization process in the region has been dominated by workers transitioning to formal status in the same type of firm. This means that the progress made in formalizing workers has happened mainly through increasing coverage of workers in small firms. 
\end{abstract}
\section{Introduction}
\begin{itemize}
    \item Motivation: Informality is a pervasive problem in the region despite many attempts to curtail it. The discussion around informality has been     \item What is informality? 
    
    The mainstream measure of informality comes from the ILO and is interested in following and comparing phenomena that are related but different. 
    \begin{itemize}
        \item Low productivity
        \item Unprotected workers
        \item Unlawful employment
    \end{itemize}
   This paper is divided in two sections the first section looks under the hood of the mainstream informality definition, studies the evolution of alternative measures trying to characterize different segments of informality, document the progress made in reducing informality in the region in the last two decades and their different margins. The second section goes over a summary of the reforms that have been implemented in the region with the goal of reducing informality. 
\end{itemize}
\section{Complexity of Informality}
\begin{itemize}
    \item Mention the history of the term and related literature: Harris Todaro 1970, De Soto, Perryet. al. Ulyssea 2017
 
    \item The mainstream definition of informality comes from the ILO (include the matrix showing the definition of informality). Check the documents coming out of the statistical meeting where they define informality and try to get the concerns behind. 
    \item Include a disaggregation of the different types of ILO informality ( prior: social security explains most of the informal)
       \item International organization have produce many flagship reports on informality : incluir OECD, WB 2021 shadow economy, ILO , IDB, etc. 
       \item the regional discussion has constantly portrayed informality as a public enemy.  
    \item In the next section we look separately at the different components of informality and relate each component with the main policy concern that explains...
\end{itemize}

\section{Labor Market Structure of the Average LAC Country}
\todo[inline] {Include figures for the average LAC country i.e. simple average of the country level measures the min and the max/ (bar=mean dots=min and max) for the latest year}
Figure \ref{fig:lab_mkt_str} shows the structure of the labor market for the average latin american country. 
\begin{itemize}
    \item Structure of the labor market 
    \begin{itemize}
        \item Prime age (25-54)$/$ PT
        \item (18-65)$/$PT
        \item \% of LF with Higher education
        \item Participacion [18-65] $/$18-65
        \item Participacion Femenina [18-65] $/$females 18-65
        \item Desempleo [18-65]$/$ PEA \& 18-65
        \item Desempleo Femenino [18-65]$/$ female PEA \& 18-65
\begin{figure}[!htb]
\centering
  \caption{Structure of labor market last year available}
\begin{subfigure}{.5\textwidth}
  \centering
  \includegraphics[width=1\linewidth]{latex/figures/Snapshot/Structure of labor market_a.png}
  \label{fig:labmarket1}
\end{subfigure}%
\begin{subfigure}{.5\textwidth}
  \centering
  \includegraphics[width=1\linewidth]{latex/figures/Snapshot/Structure of labor market_b.png}
  \label{fig:labmarket2}
\end{subfigure}
\label{fig:test}
\footnotesize{Source:Household Surveys-SEDLAC.}
\footnotesize{Note: Latin American simple average for last year available. Some LAC countries don’t have information for 2021, in that cases we use the last available year, Chile 2022, Guatemala 2014; Honduras 2019; Mexico 2018 and Uruguay 2019.}

\end{figure}
        
    \end{itemize}
    \item Structure of Employment
    
    \begin{itemize}
        \item Salaried
        \item Salaried- Small Firm
        \item Salaried -Government
        \item Self-employed 
        \item Self-employed secondary+
        \item Self-employed- no employees
        \item Self-employed- small firm
        \begin{figure}[!htb]
        \centering
        \caption{Structure of employment last year available}     
        \includegraphics[scale=.3]{latex/figures/Snapshot/Structure of employment.png}
        \label{fig:employment}
        \footnotesize{Source:Household Surveys-SEDLAC.}
        \footnotesize{Note: Latin American simple average for last year available. Some LAC countries don’t have information for 2021, in that cases we use the last available year, Chile 2022, Guatemala 2014; Honduras 2019; Mexico 2018 and Uruguay 2019.}
        \end{figure}
         \todo[inline]{Check the zeros- in figure \ref{fig:firmsize} Do you have zeroes because the aggregation is not working properly? Use as many size categories as possible but keep in mind that all countries should have a way to get the share of workers in each of the aggregate size categories. }
         
         \begin{figure}[!htb]

        \centering
        \caption{Structure of self-employment last year available}     
        \includegraphics[scale=.3]{latex/figures/Snapshot/Structure of self employment.png}
        \label{fig:selfemployment}
        \footnotesize{Source:Household Surveys-SEDLAC.}
        \footnotesize{Note: Latin American simple average for last year available. Some LAC countries don’t have information for 2021, in that cases we use the last available year, Chile 2022, Guatemala 2014; Honduras 2019; Mexico 2018 and Uruguay 2019.}
        \end{figure}
\todo[inline]{Make sure to explain if the categories are exclusive. That is do: Employers -5 less includes self employed with higher education? }
         
        
    \end{itemize}
\item Structure of Employment II - Sector and Contribution to SS

\begin{figure}[!htb]
        \centering
        \caption{Structure of employment by sector last year available}     
        \includegraphics[scale=.3]{latex/figures/Snapshot/Structure of employment and sector.png}
        \label{fig:sector}
        \footnotesize{Source:Household Surveys-SEDLAC.}
        \footnotesize{Note: Latin American simple average for last year available. Some LAC countries don’t have information for 2021, in that cases we use the last available year, Chile 2022, Guatemala 2014; Honduras 2019; Mexico 2018 and Uruguay 2019.}
\end{figure}

        
\item Employment by Firm Size
\begin{figure}[!htb]
        \centering
        \caption{Structure of employment by firm size last year available}     
        \includegraphics[scale=.3]{latex/figures/Snapshot/Structure of employment by firm size.png}
        \label{fig:firmsize}
        \footnotesize{Source:Household Surveys-SEDLAC.}
        \footnotesize{Note: Latin American simple average for last year available. Some LAC countries don’t have information for 2021, in that cases we use the last available year, Chile 2022, Guatemala 2014; Honduras 2019; Mexico 2018 and Uruguay 2019.}
        \end{figure}
        
 \item Social security contributions
    \begin{itemize}
        \item Does not Contribute (DNC) 
        \item DNC- secondary+
        \item DNC-Salaried
        \item DNC-Salaried -small firms
        \item DNC-Self employed
        \item ILO- informality
        \begin{figure}[!htb]
        \centering
        \caption{Social security non-contributions rate by group last year available}     
        \includegraphics[scale=.3]{latex/figures/Snapshot/Social security contributions.png}
        \label{fig:SScontributions}
        \footnotesize{Source:Household Surveys-SEDLAC.}
        \footnotesize{Note: Latin American simple average for last year available. Some LAC countries don’t have information for 2021, in that cases we use the last available year, Chile 2022, Guatemala 2014; Honduras 2019; Mexico 2018 and Uruguay 2019.}
        \end{figure}
\todo[inline]{make sure that in figure \ref{fig:SScontributions} the categories are not exclusive and add to the note "The groups are not exclusive. "}
        
    \end{itemize}


\item Life cycle- self employment and salaried informal
 \begin{figure}[!htb]
        \centering
        \caption{Age profile last year available}     
        \includegraphics[scale=.3]{latex/figures/Snapshot/age_profile.png}
        \label{fig:age_pro}
        \footnotesize{Source:Household Surveys-SEDLAC.}
        \footnotesize{Note: Latin American simple average for last year available. Some LAC countries don’t have information for 2021, in that cases we use the last available year, Chile 2022, Guatemala 2014; Honduras 2019; Mexico 2018 and Uruguay 2019.}
        \end{figure}
\item Decomposition of changes in the informality rate

\end{itemize}

     

     
     
\section{Dynamics: last 20 years}  
 

\subsection{Regional figures}

\begin{figure}[!htb]
\centering
  \caption{Evolution of alternative informality related measures}
\begin{subfigure}{.5\textwidth}
  \centering
  \includegraphics[width=1\linewidth]{latex/figures/Series/LAC_ipo_fig1.png}
  \label{fig:sub1}
\end{subfigure}%
\begin{subfigure}{.5\textwidth}
  \centering
  \includegraphics[width=1\linewidth]{latex/figures/Series/LAC_ipo_fig2.png}
  \label{fig:sub2}
\end{subfigure}
\label{fig:test}
\footnotesize{Source: Household Surveys-SEDLAC and ILO.}

\end{figure}


\subsection{Cross country figures}
\subsubsection{Household}
\begin{figure}[!htb]
    \centering
     \caption{Snapshot of LAC’s household contribution status -last available year}     \includegraphics[scale=.3]{latex/figures/Household/snapshot_household.png}
    \label{fig:Householdlastyear}
    \footnotesize{Note: Some countries don’t have information for 2021, in that cases we use the last available year, Chile 2017, Guatemala 2014; Honduras 2019; Mexico 2018 and Uruguay 2019.}
\end{figure}

\begin{figure}[!htb]
    \centering
     \caption{Snapshot of LAC’s household employment condition for 2005-2021}     
     \includegraphics[scale=.3]{latex/figures/Household/snapshot_household_2005-2021.png}
    \label{fig:Household20052021}
    \footnotesize{Note: Some countries don’t have information for the 2005 or 2020, in those cases we are using the closest year, for Bolivia 2002; Chile 2006 and 2017; Colombia 2008; Guatemala 2004 and 2014; Mexico 2006 and 2018; Honduras 2019 and Uruguay 2019.}
 
        
    \footnotesize{Note: The figure reports the household contribution status to social security of households with at least one worker.   All contribute: corresponds to the percentage of households where all workers contribute to SS. Some contribute: corresponds to the percentage of households where some workers contribute to SS but not all. DNC – has partner: corresponds to the percentage of households where any worker contributes to SS, but the head of household have a partner. DNC – no partner: corresponds to the percentage of households where any worker contributes to SS, but the head of household do not have a partner. }
\end{figure}

\subsubsection{Individual level} %De este nombre no estoy segura

\begin{figure}[!htb]
    \centering
     \caption{Salaried who don’t contribute to SS}     
     \includegraphics[scale=.3]{latex/figures/Snapshot/snapshot_informal_ss_dep.png}
    \label{fig:SalariedSS}
    \footnotesize{Note: Some countries don’t have information for the chosen years, in that cases we are using the closest year, for Bolivia 2002; Chile 2010; Colombia 2008; Guatemala 2004 and 2014; Mexico 2006 and 2018; Honduras 2019 and Uruguay 2019.}
\end{figure}

\begin{figure}[!htb]
    \centering
     \caption{Salaried who work at small firms}     
     \includegraphics[scale=.3]{latex/figures/Snapshot/snapshot_dependents_small.png}
    \label{fig:SalariedSmall}
    \footnotesize{Note: Some countries don’t have information for the chosen years, in that cases we are using the closest year, for Bolivia 2002; Chile 2010; Colombia 2008; Guatemala 2004 and 2014; Mexico 2006 and 2018; Honduras 2019 and Uruguay 2019.}
\end{figure}
\todo[inline]{Check footnote of figure \ref{fig:Oaxaca} and make sure the title is correct. Let's include a small appendix with the methodolgy details.}

 \begin{figure}[!htb]
        \justifying
        \caption{Microeconometric decomposition of evolution of workers who does not contribute to ss 2005-2021}     
        \includegraphics[scale=.3]{latex/figures/Snapshot/Oaxaca decomposition share.png}
        \label{fig:Oaxaca_share}
        \footnotesize{Source: Household Surveys-SEDLAC.}
        \footnotesize{Note: Each bar is the share of change in Social security contributions corresponds to the share of coefficients and the share of the endowment effects. Countries included in the sample: Argentina, Bolivia, Brazil, Chile, Colombia, Costa Rica, Dominican Republic, Ecuador, El Salvador, Guatemala, Honduras, Mexico, Panama, Peru, Paraguay and Uruguay. Some countries don’t have information for 2021, in that cases we use the last available year, for Chile 2022, Guatemala 2014; Honduras 2019; Mexico 2018 and Uruguay 2019.}
        \end{figure}






\end{document}