\documentclass[english]{article}
\usepackage[T1]{fontenc}
%\usepackage[latin9]{inputenc}
%\usepackage{geometry}
%\geometry{verbose,tmargin=1in,bmargin=1in,lmargin=1in,rmargin=1in}
\usepackage{amsmath}
\usepackage{amssymb}
\usepackage{setspace}
\usepackage[utf8]{inputenc}
\usepackage{graphicx}
\usepackage{float}
\usepackage{adjustbox}
\usepackage{gensymb}
\usepackage{amssymb}
\usepackage{array}
\usepackage{ragged2e}
\usepackage{lipsum}
\onehalfspacing
\usepackage{babel}


\begin{document}
\title{Informality Trend in LAC: Small and Disappointing Progress }
\maketitle
\begin{abstract}
    This descriptive paper uses household and employment surveys from the region to paint a more complete picture of the different aspects of informality. We show that there has been progress on increasing the share of dependent workers who contribute to social security but that the productive structure of the economies have remained unchanged. We argue that this is mainly coming from the focus of governments policies on facilitating the registration of low productivity workers, therefore subsidizing low productivity firms. To make this point we use two approaches: a qualitative approach based on the analysis of policies that have been identified as successful in reducing informality. Second, we use a simple decomposition on different formalization margins to show that the formalization process in the region has been dominated by workers transitioning to formal status in the same type of firm. This means that the progress made in formalizing workers has happened mainly through increasing coverage of workers in small firms. 
\end{abstract}
\section{Introduction}
\begin{itemize}
    \item Motivation: Informality is a pervasive problem in the region despite many attempts to curtail it. 
    \item The mainstream measure of informality comes from the ILO and includes many dissimilar phenomena. 
    \begin{itemize}
        \item Low productivity and unprotected workers
    \item There is a common understanding on the fact that informality is bad
    \item 
    \end{itemize}
\end{itemize}

\begin{figure} [H]
    \raggedright
    \caption{La proporción de mexicanos que tiene una mala percepción sobre los inmigrantes es más alta que la del país promedio de AL}
    \subcaption*{\textit{Porcentaje en cada categoría de respuesta (\%)}}
    \includegraphics[scale=0.4]{FIGURAS CH3/MX_percep1.pdf}
    
    \footnotesize{\note{Fuente: Latinobarómetro.}}
    \label{fig:MX_percep1}
\end{figure}

\end{document}